Transformers form a core part of modern technology, being used for impedance matching within electronics and 
electricity transmission on the grid. In both systems, the efficiency and performance are determined by how the 
transformer responds to an applied magnetic field. 
$\\$
$\\$
This response is described by the relationship between the 
magnetic flux density $B$ and the magnetising field $H$, known as the B–H curve. By altering the core of the 
transformer its properties can be changed. For ferromagnetic materials, the B–H curve is non-linear and exhibits 
hysteresis ($\textbf{\Cref{fig:hysteresis_curve_example}}$). $\\$

\begin{figure}[h]
    \centering
    \includegraphics[width=0.6\textwidth]{../report/figures/Keim_11_12_Fig_1.jpg}
    \caption{A typical hysteresis curve for a ferromagnetic material, showing the relationship between $B$ and $H$. 
    \cite{hysteresis_curve_example}.}
    \label{fig:hysteresis_curve_example}
\end{figure}



$\\$
The most efficient transformers are made by having a high magnetic susceptibility ($\mu_r$). In section 2 of 
this report we will go through the relevant theory. Then the experimental setup in section 3 followed by the 
results it has yielded in section 4 penultimately the analysis of these and their comparison against theoretical
 expectations in section 5 and finally the conclusion.
