Transformers form a core part of modern technology, being used for impedance matching within electronics and 
electricity transmission on the grid. In both systems, the efficiency and performance are determined by how the 
transformer responds to an applied magnetic field. 
$\\$
$\\$
This response is described by the relationship between the 
magnetic flux density $B$ and the magnetising field $H$, known as the $B$-$H$ curve. By altering the core of the 
transformer its properties can be changed. For ferromagnetic materials, the $B$-$H$ curve is non-linear and exhibits 
hysteresis ($\textbf{\Cref{fig:hysteresis_curve_example}}$) due to magnetic subdomains in the material.

\begin{figure}[h]
    \centering
    \includegraphics[width=0.6\textwidth]{../report/figures/Keim_11_12_Fig_1.jpg}
    \caption{A typical hysteresis curve for a ferromagnetic material, showing the relationship between $B$ and $H$. 
    \cite{hysteresis_curve_example}.}
    \label{fig:hysteresis_curve_example}
\end{figure}


$\\$
The most efficient transformers are made by having a high magnetic susceptibility ($\mu_r$) and low energy 
loss per cycle. This experiment aims to quantify the usability of different materials as transformer cores,
by measuring the energy loss per cycle per unit volume (given by the area of a hysterysis loop) and the maximum 
magnetic susceptibility (given by the gradient).
Section 2 of this report will go through the relevant theory. Section 3 outlines the experimental setup, and 
Section 4 shows the results obtained. This data is analysed and compared against theoretical expectations in 
Section 5. The overall conclusions are then presented in Section 6. 


