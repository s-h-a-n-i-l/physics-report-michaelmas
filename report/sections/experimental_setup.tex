\subsection{Coil assembly and primary circuit}
Two coils, one of  $ \sim 500$ turns and one of $ \sim 400$ turns, were co-axially wound into a solenoid.
The coil with 500 turns was designated the primary coil and connected in series with 
a $2 \pm 0.1 \,\Omega$ resistor ($R_p$) to reduce the current. This allowed us to calculate the 
current through the coil and thus the magnetic field strength through the coil 
($I = \frac{V}{R_p} \rightarrow H = \frac{nI}{L}\,\, (eqn \ref{H eqn})$) and ensured the coil did not overheat. 
This part of the circuit was completed by connecting it to a signal generator set to 
produce an AC voltage of $5 \pm 0.01 V_{pp}$  (Peak to Peak) at $50 \pm 0.1 \, \text{Hz}$.
$\\$
$\\$
The secondary coil was then attached to an integrator circuit.

\begin{figure}[h!]
    \centering
    \includegraphics[width=0.75\textwidth]{../report/figures/full circuit.png}
    \caption{integrator circuit.}
    \label{fig:full_circuit}
\end{figure}
$\\$
$\\$

\subsection{Integrator design and calibration}
The integrator circuit was set up as shown below:
\begin{figure}[h!]
    \centering
    \includegraphics[width=0.5\textwidth]{../report/figures/intergrator_circuit.png}
    \caption{integrator circuit\cite{transformer_coils}.}
    \label{fig:integrator_circuit}
\end{figure} $\\$
$R_2$ was chosen to be, 1M$\Omega$. $\\$ 
After measuring $R_2$, it was found to be 991.2$\pm 0.1 K\Omega$
$\\$
The value of $C_1$ s chosen such that $    Z_i << 1 M \Omega $ at 50 Hz.
$\\$
\begin{gather}
    Z_i << 1 M \Omega 
    \\
    Z_i  = \frac {1}{\omega C_1}
    \\
    \frac {1}{\omega C_1} << 1 M \Omega
    \\
    \frac {1}{2 \pi \times 50 \times C_1} << 10^6
    \\
    C_1 >> \, 3.18 nF
\end{gather} 
Therefore Picking 100 nF was suitable.
$\\$
Next $R_i$ needed to be chosen. To make the calculations easier, an $R_i$ value that would give a gain of 1 was chosen.
\begin{gather}
    gain = \abs{\frac{V_{out}}{V_{in}}} =  \frac {1}{\omega R_i C_1}
    \\
    1 = \frac {1}{\omega R_i C_1}
    \\
    R_i = 31.83 K \Omega
    \\
    R_i \approx 30 K \Omega
\end{gather}
the chosen resistor and capacitor were then measured accurately, Giving:
$\\$
\begin{gather}
    R_i = 32,830 \pm 5 \Omega \\
    C_1 = 97.8 \pm 0.1 nF \\
    \therefore \nonumber\\
    theoretical \,\, gain = \frac {1}{2 \pi \times 50 \times 32,830 \times 97.8 \times 10^{-9}} = 0.9913
\end{gather}
$\\$
After setting up the circuit, the oscilloscope was attached to the signal generator to measure the input and to 
measure the output of the circuit.
\begin{figure}[h!]
    \centering
    \includegraphics[width=0.75\textwidth]{../report/figures/integrator_channels.png}
    \caption{Measurements of the input and output of the circuit figure.}
    \label{fig:gain_measurement}
\end{figure}
$\\$
\begin{equation}
    Measured \,\, gain = \abs{\frac {V_{out}}{V_{in}}} = \frac {14.38 V}{10.49 V} = 1.37 \pm 0.005
\end{equation}
$\\$
This is a $38\%$ deviation from the theoretical value and may be due to a square wave being tested. 
This is composed of sinusoidal waves of many frequencies and so will have some at the odd harmonics 
of the fundamental frequency of the integrator circuit (e.g. 3f = 150 Hz, 5f = 250 Hz, etc.). This is combined 
with the op-amp not being ideal and having a limited bandwidth can result in a different gain to the theoretical value.
This is treated as a systematic error in the experiment and was used as gain in the subsequent calculations.
$\\$

\subsection{Data acquisition and sample preparation}

The samples used were:$\\$
$\bullet$ Mild steel$\\$
$\bullet$ Transformer iron$\\$
$\bullet$ CuNi alloy at 10$^\circ$C and above 40$^\circ$C$\\$
$\\$
the coil length and the cross-sectional area of the samples were measured.
This was done using a vernier caliper to measure the length and width of the cuboid mild steel and transformer iron samples 
and the diameter of the cylindrical CuNi alloy sample.
$\\$
Then the experiment was run:
The signal generator was turned on to feed an AC current into the primary coil
after Waiting until the circuit reached a steady state the data points for 20 cycles were saved.
This was done immediately to ensure the coil didn't heat up and change its resistance during the measurement.
The coil was then turned off and this was repeated for the other samples.
$\\$
$\\$
First the measurements were performed without a sample (air) to act as a control, then metals.
$\\$
$\\$
When measuring the CuNi alloy it was placed into hot water to raise its temperature to above 40$^\circ$C
and then placed into ice water to cool it to 10$^\circ$C.
The low time difference between the samples being removed from the water and being tested results in minor temperature 
changes during the measurement but the actual temperature was not measured during the hysteresis and thus the temperatures should be taken as approximate.
$\\$
$\\$
Voltages across the series 2 $\Omega$ resistor (Channel A) and integrator output (Channel B) were recorded using a PicoScope 
at a sampling rate of 1.95 MHz. For each sample 20 cycles at 50 Hz ($\sim$0.32 s) were recorded. Was then exported to be analysed using a Python script that splits up 
the individual loop and plots them for a visual check during the experiment and allowed for further analysis later in Section \ref{analysis}.
$\\$